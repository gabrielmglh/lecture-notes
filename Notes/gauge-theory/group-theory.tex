\documentclass{article}

%%%%%%%%PACKAGES%%%%%%%%
\usepackage{amsmath} 
\usepackage{mathtools}
\usepackage{amsfonts}
\usepackage{siunitx} 
\usepackage{physics}
\usepackage{geometry}
\usepackage{graphicx}
\usepackage{float}
\usepackage{hyperref}
\usepackage{url}
\geometry{margin=1in} 

\newtheorem{definition}{Definition}

%%%%%%%%TITLE%%%%%%%%
\title{Introdução à Teoria de Grupos}
\author{Gabriel C. Magalhães\footnote{gabriel.capelini@uel.br}}
\date{2025}

\begin{document}
\maketitle

\begin{abstract}
Essas notas foram escritas como material de apoio para um minicurso ministrado na XXIX Semana da Física UEL e são baseadas principalmente em \cite{jakob}. 
\end{abstract}

\tableofcontents
\pagebreak

\section{Introdução}
\subsection{Simetrias}

\subsection{Definição de grupo}
Um grupo é um conjunto $G$ com um mapa $cdot$ que satisfaz as seguintes propriedades:  
\begin{itemize}
	\item \textbf{Fechamento:} Para todo $g,g'\in G,\; g\cdot g'\in G.$ 
	\item \textbf{Identidade:} Existe um elemento $e\in G$ tal que para todo $g\in G$, $g\cdot e=e\cdot g = g$. Chamamos esse elemento de \textbf{identidade. }
	\item \textbf{Elemento inverso}: Para todo $g\in G$, existe $g'\in G$ tal que $g\cdot g'=g'\cdot g = e$. Chamamos esse elemento de \textbf{elemento inverso} e o denotamos por $g'\equiv g^{-1}$. 
	\item \textbf{Associatividade:} Para todo $g_1,g_2,g_3\in G,(g_1\cdot g_2)\cdot g_3=g_1\cdot(g_2\cdot g_3).$
\end{itemize}
\section{Rotações em duas dimensões}
\subsection{Números complexos}
\section{Rotações em três dimensões}
\subsection{Quaternions}
\section{Álgebra de Lie}
\subsection{Geradores}
\subsubsection{Geradores do grupo SO(3)}
\subsubsection{Geradores do grupo SU(2)}
\section{Teoria de Representação}
\subsection{Representações do grupo SU(2)}
\section{Grupo de Lorentz}


\pagebreak
\nocite{*}
\bibliography{refs.bib}
\bibliographystyle{ieeetr}
\end{document}
