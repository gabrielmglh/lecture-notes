\documentclass{article}

%%%%%%%%PACKAGES%%%%%%%%
\usepackage{amsmath} 
\usepackage{siunitx} 
\usepackage{physics}
\usepackage{geometry}
\geometry{margin=1in} 

%%%%%%%%TITLE%%%%%%%%
\title{Your Article Title}
\author{Your Name}
\date{\today}

\begin{document}

\maketitle

\section{Introduction}
This is the introduction to your article.

\section{Equations}
Here's an example of using physical symbols and equations:

The speed of light in a vacuum, \(c\), is approximately \SI{3e8}{\meter\per\second}. 
The famous equation, \(E=mc^2\), relates energy (\(E\)), mass (\(m\)), and the speed of light (\(c\)).

\begin{equation}
    E = mc^2
\end{equation}

You can also use the \texttt{physics} package to simplify notation, like this:

\begin{equation}
    \vb{F} = m \vb{a}
\end{equation}

\section{Conclusion}
This is the conclusion of your article.

\end{document}
